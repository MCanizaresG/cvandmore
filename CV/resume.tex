% This template is designed to offer an aesthetically pleasing resume that adheres to a formal and institutional tone, making it suitable for applications to companies and research centers requiring a high degree of professionalism. Navy blue has been chosen as the primary color to align with these objectives.
% The code is well-documented and annotated, allowing users to easily customize and modify it according to their needs. Please note that the template's content is meant to be humorous and should not be taken literally. We are grateful for your interest in using this template for your professional endeavors.
% Author: Christian Maria Giannetti

%----------------------------------------------------------------------------------------
%  Packages And Other Document Configurations
%----------------------------------------------------------------------------------------

\documentclass{resume} % Use the custom resume.cls style

% Document margins
\usepackage[left=0.75in,top=0.6in,right=0.75in,bottom=0.6in]{geometry}

% Color and hyperlink packages
\usepackage{xcolor}
\usepackage{hyperref}

% Footnote and margin adjustment packages
\usepackage{footnote}
\usepackage{changepage}

% Fontawesome package for icons
\usepackage{fontawesome}

% Tabularx package for custom tables
\usepackage{tabularx}

% Define navyblue color
\definecolor{navyblue}{RGB}{0,54,123}

%----------------------------------------------------------------------------------------
%   Customizations
%----------------------------------------------------------------------------------------

% Define italicitem, bolditem, and plainitem commands
\newcommand{\italicitem}[1]{\item{\textit{#1}}}
\newcommand{\bolditem}[1]{\item{\textbf{#1}}}
\newcommand{\plainitem}[1]{\item{#1}}

% Define user-friendly link command for hyperlinks
\newcommand{\link}[2]{{\href{#1}{#2}}}

\newcommand{\entry}[2]{#1 & #2 \tabularnewline} % Defines an entry with two arguments: #1 for the first column and #2 for the second column

%----------------------------------------------------------------------------------------
%   Define envsection command for defining a new environment section
%----------------------------------------------------------------------------------------

\newcommand{\tableEnv}[2]{%
  \begin{rSection}{#1} % Begin rSection with the given name
    \begin{adjustwidth}{0.0in}{0.1in} % Set the left and right margins
      \begin{tabularx}{\linewidth}{@{} >{\bfseries}l @{\hspace{6ex}} X @{}}
        #2 % Print the content inside the tabularx environment
      \end{tabularx}
    \end{adjustwidth}
  \end{rSection}
}

%----------------------------------------------------------------------------------------
%   Begin document
%----------------------------------------------------------------------------------------

% Set name with navyblue color
\name{Manuel Ca\~{n}izares}



\begin{document}


\printPersonalInfo{
  \personalInfo{\info{Avda. Mazarredo, 14. 48003 Bilbao, Spain.}}
  \personalInfo{\info{\href{mailto:manuel.canizares@ricam.oeaw.ac.at}{manuel.canizares@ricam.oeaw.ac.at}} \infoSeparator\info{\href{https://sites.google.com/view/mcanizares/}{Personal Site}}}
  \personalInfo{\tag{Place of birth}\info{Chiclana de la Frontera, Spain} \infoSeparator\tag{Date of birth}\info{03/02/1996}}
  \personalInfo{\info{Updated \today}}
}

%----------------------------------------------------------------------------------------
%   Work experience section
%----------------------------------------------------------------------------------------

\begin{rSection}{Positions}

    % First work experience entry
    \begin{rSubsection}{RICAM (Johann Radon Institute for Computational and Applied Mathematics) / OEAW (Austrian Academy of Sciences)}{Novemeber 2024 - now}{Postdoctoral researcher}{Linz, Austria}
        \item Working under the supervision of Otmar Scherzer.
    \end{rSubsection}
    \begin{rSubsection}{BCAM (Basque Center for Appplied Mathematics)}{September 2020 - October 2024}{PhD Student}{Bilbao, Spain}
        \item Working under the supervision of Pedro Caro in the project \textit{Interplays between Harmonic Analysis and Inverse Problems}, with a grant from the Spanish Research Agency (AEI), with reference \textit{PRE2019-091776}
    \end{rSubsection}
  
    
\end{rSection}


%----------------------------------------------------------------------------------------
%   Education section
%----------------------------------------------------------------------------------------

\begin{rSection}{Education}

    % Master's degree entry
    \begin{rSubsectionNoBullet}{\bf PhD in Mathematics}{2024}{Doctoral program in mathematics and statistics}{UPV/EHU}
        \italicitem{Title: Identifying quantum hamiltonians in the presence of electric interactions. An analytic approach.}
        \italicitem{Grade: Cum Laude}
        \italicitem{Supervisor: Pedro Caro}
        \italicitem{Tribunal: Alberto Ruiz, Luz Roncal and Tony Liimatainen}
    \end{rSubsectionNoBullet}
    
    % Bachelor's degree entry
    \begin{rSubsectionNoBullet}{\bf Master's degree in Mathematical Physics}{2020}{Master's program FISyMAT}{Universidad de Granada}
        \italicitem{Final grade: 8.72/10 - 2.6/4}
        \italicitem{Master's thesis: Synchronization and Aggregation models. The swarmallators model.}
        \italicitem{Supervisor: Juan Soler}
    \end{rSubsectionNoBullet}
    
    % High School diploma entry
    \begin{rSubsectionNoBullet}{\bf Bachelor's degree in Physics}{2018}{Double bachelor's program in physics and mathematics}{Universidad de Sevilla}
        \italicitem{Final grade: 7.32/10 - 1.83/4}
        \italicitem{Bachelor's thesis: Width of Interfaces in the 2-Dimensional Ising Model.}
        \italicitem{Supervisor: Gernot M\"unster (written during my Erasmus stay at WWU M\"unster, Germany)}
    \end{rSubsectionNoBullet}

    \begin{rSubsectionNoBullet}{\bf Bachelor degree in Mathematics}{2018}{Double bachelor's program in physics and mathematics}{Universidad de Sevilla}
        \italicitem{Final grade: 7.86/10 - 2.05/4}
        %\italicitem{Bachelor thesis: Width of Interfaces in the 2-Dimensional Ising Model.}
        %\italicitem{Supervisor: Gernot M\"unster (written during my Erasmus stay at WWU M\"unster, Germany)}
    \end{rSubsectionNoBullet}

\end{rSection}


\begin{rSection}{Publications}
    \begin{rSubsectionNoBullet}{Local near-field scattering data enables unique reconstruction of rough\\ electric potentials}{2024}{Inverse Problems \textbf{\textrm{40}} \textrm{065004}}{\href{https://doi.org/10.1088/1361-6420/ad3eaa}{DOI},\href{https://arxiv.org/abs/2311.09036}{arXiv}}
        \item Manuel Ca\~nizares
    \end{rSubsectionNoBullet}
    \begin{rSubsectionNoBullet}{Interface Roughening in Two Dimensions}{2021}{Journal of Statistical Physics, \textbf{182}, 61}{\href{https://doi.org/10.1007/s10955-021-02738-w}{DOI},\href{https://arxiv.org/abs/2004.13807}{arXiv}}
        \item Gernot M\"unster \& Manuel Ca\~nizares
    \end{rSubsectionNoBullet}
    
\end{rSection}

\begin{rSection}{Talks}
    \begin{rSubsectionNoBullet}{Indentifying electric potentials via the local
        near-field scattering pattern\\ at fixed energy}{2024}{Inverse Problems and Mathematical Imaging  group seminar}{RICAM. Linz, Austria}
        \italicitem {Seminar talk}
    \end{rSubsectionNoBullet}
    \begin{rSubsectionNoBullet}{Indentifying electric potentials via the local
        near-field scattering pattern\\ at fixed energy}{2024}{9th European Congress of Mathematics. Minisymposium: Analytical, computational and geometrical approaches to inverse problems.}{Sevilla, Spain}
        \italicitem {Invited mini-symposium talk}
    \end{rSubsectionNoBullet}
    \begin{rSubsectionNoBullet}{Indentifying electric potentials via the local
        near-field scattering pattern\\ at fixed energy}{2024}{IV Mathematical Analysis Days}{Universidad de La Rioja. Logro\~no, Spain}
        \italicitem {Invited talk}
    \end{rSubsectionNoBullet}
    \begin{rSubsectionNoBullet}{Indentifying electric potentials via the local
        near-field scattering pattern\\ at fixed energy}{2024}{Seminari d'Analisi de Barcelona}{UPC, UB and UAB. Barcelona, Spain}
        \italicitem {Seminar talk}
    \end{rSubsectionNoBullet}
    \begin{rSubsectionNoBullet}{Determination of delta-shell and critically-singular potentials
        with local \\near-field scattering data}{2023}{HAPDEGMT}{UPV/EHU. Bilbao, Spain}
        \italicitem {Short talk}
    \end{rSubsectionNoBullet}
    \begin{rSubsectionNoBullet}{Determination of delta-shell and critically-singular potentials
        with local \\near-field scattering data}{2022}{Inverse Days}{FIPS and University of Eastern Finland. Kuopio, Finland}
        \italicitem {Short talk}
    \end{rSubsectionNoBullet}
    
\end{rSection}

%----------------------------------------------------------------------------------------
%   Extracurricular activities section
%----------------------------------------------------------------------------------------
%
%\begin{rSection}{Attended courses, conferences and other activities}
%
%% First extracurricular entry
%\begin{rSubsectionNoBullet}{Procrastinators Anonymous: Machine Learning for Laziness}{November 2015}{Successfully delayed attending the course}{Rome, Italy}
%    \plainitem{Developed an AI that predicts the perfect moment to start binge-watching Dr. House episodes instead of working on important tasks.}
%\end{rSubsectionNoBullet}
%
%% Third extracurricular entry
%\begin{rSubsectionNoBullet}{Dumb Literature Club: Famous Books with Ridiculous Twists}{June 2014 - Present}{Proud Member}{Popejoy, Iowa}
%    \plainitem{Regularly participate in discussions concerning famous literature works, such as "The Great Catsby", which is basically the Great Gatsby, but with cats.}
%\end{rSubsectionNoBullet}
%
%\end{rSection}
%
%----------------------------------------------------------------------------------------
% Technical skills section
%----------------------------------------------------------------------------------------

\begin{rSection}{Grants and Scholarships}
    \begin{rSubsectionNoBullet}{Aid for pre-doc contracts for the training of doctors 2019}{2020-2024}{AEI (Agencia Estatal de Investigaci\'on)}{Spain}
        \italicitem {Reference: PRE2019-091776}
    \end{rSubsectionNoBullet}
    \begin{rSubsectionNoBullet}{Erasmus+ Scholarship}{2017-2018}{}{}
       \vspace{-.3cm} \italicitem {Stay at WWU M\"unster, Germany}
    \end{rSubsectionNoBullet}
    \begin{rSubsectionNoBullet}{Scholarship for Undergraduate studies}{2013-2020}{Spanish Ministry of Education}{Spain}
        \italicitem {\vspace{-.3cm}}
    \end{rSubsectionNoBullet}
\end{rSection}

\tableEnv{Technical skills}{
    \entry{Programming Languages/Tools}{C, C++, Java, Python, \LaTeX, Matlab, Haskell}
}

%----------------------------------------------------------------------------------------
% Language proficiencies section
%----------------------------------------------------------------------------------------

\tableEnv{Language proficiencies}{
    \entry{Spanish}{Native}
    \entry{English}{Fluent. CAE degree by Cambridge (C1) obtained in 2012}
    \entry{Italian}{Medium level}
    \entry{French}{Basic level}
}

\tableEnv{References}{
    \entry{Pedro Caro. PhD supervisor}{\href{mailto:pcaro@bcamath.org}{pcaro@bcamath.org}}
    \entry{Ioannis Parissis. Collaborator}{\href{mailto:ioannis.parissis@gmail.com}{ioannis.parissis@gmail.com}}
    \entry{Juan Soler. Master's thesis supervisor}{\href{mailto:jsoler@ugr.es}{jsoler@ugr.es}}
    \entry{Gernot M\"unster. Bacherlor's thesis supervisor}{\href{mailto:munsteg@uni-muenster.de}{munsteg@uni-muenster.de}}


}

\end{document}