%%%%%%%%%%%%%%%%%%%%%%%%%%%%%%%%%%%%%%%%%
% Long letter 南京大学版 Nanjing University Version
% Version 1.0 (2022-03-27)

% This template was revised by Zheng-hu Nie(brian.nie@gmail.com) based on Fanchao Chen (chenfc@fudan.edu.cn).
% This template originates from:
% https://www.LaTeXTemplates.com

%%%%%%%%%%%%%%%%%%%%%%%%%%%%%%%%%%%%%%%%%

%----------------------------------------------------------------------------------------
%	PACKAGES AND OTHER DOCUMENT CONFIGURATIONS 文档基础配置
%----------------------------------------------------------------------------------------

\documentclass{article}

%\usepackage{charter} % Use the Charter font

\usepackage[
	a4paper, % Paper size
	top=1in, % Top margin
	bottom=1in, % Bottom margin
	left=1in, % Left margin
	right=1in, % Right margin
	%showframe % Uncomment to show frames around the margins for debugging purposes
]{geometry}

\setlength{\parindent}{0pt} % Paragraph indentation
\setlength{\parskip}{1em} % Vertical space between paragraphs

\usepackage{graphicx} % Required for including images

\usepackage{hyperref}

\usepackage{fancyhdr} % Required for customizing headers and footers

\fancypagestyle{firstpage}{%
	\fancyhf{} % Clear default headers/footers
	\renewcommand{\headrulewidth}{0pt} % No header rule
	\renewcommand{\footrulewidth}{1pt} % Footer rule thickness
}

\fancypagestyle{subsequentpages}{%
	\fancyhf{} % Clear default headers/footers
	\renewcommand{\headrulewidth}{1pt} % Header rule thickness
	\renewcommand{\footrulewidth}{1pt} % Footer rule thickness
}

\AtBeginDocument{\thispagestyle{firstpage}} % Use the first page headers/footers style on the first page
\pagestyle{subsequentpages} % Use the subsequent pages headers/footers style on subsequent pages

%----------------------------------------------------------------------------------------

\begin{document}

%----------------------------------------------------------------------------------------
%	FIRST PAGE HEADER
%----------------------------------------------------------------------------------------


\vspace{-1em} % Pull the rule closer to the logo

\rule{\linewidth}{1pt} % Horizontal rule

%\bigskip\bigskip % Vertical whitespace

%----------------------------------------------------------------------------------------
%	YOUR NAME AND CONTACT INFORMATION
%----------------------------------------------------------------------------------------

\hfill
\begin{tabular}{l @{}}
	\hfill \today \bigskip                                              \\ % Date
	\hfill Manuel Ca\~{n}izares                                         \\
	\hfill \href{mailto:mcanizares@bcamath.org}{mcanizares@bcamath.org} \\
	\hfill BCAM (Basque Center for Applied Mathematics)                 \\
	\hfill 48003. Bilbao, Spain                                         \\ % Address
\end{tabular}

\bigskip % Vertical whitespace

%----------------------------------------------------------------------------------------
%	LETTER CONTENT
%----------------------------------------------------------------------------------------


\textbf{RELATIONSHIP BETWEEN THE MANUSCRIPT AND THE PUBLISHED PAPER}

\begin{itemize}
	\item \textbf{Section 2} of the paper (solution of the direct problem) became \textbf{Chapter 4}. I expanded a little bit on explaining the free resolvent and the Neumann series argument, and gave some more context, but the mathematical content is quite the same.
	\item \textbf{Section 3} (orthogonality relation) and \textbf{Section 4} (CGO solutions and proof of the identifiability result) are now together in \textbf{Chapter 6} of the thesis. The only noticeable change is that the perturbation of the Carleman estimate migrated to a separate chapter: \textbf{Chapter 3}. There, I also state a couple of unique continuation properties that are derived from it. I decided to to this because the contents of Chapter 3 will be used in Chapters 4, 5 and 6. I found it make more sense to do this this way.
	\item \textbf{Appendix A} of the paper (layer potentials) is now a section inside of \textbf{Chapter 5} of the thesis. You will find some changes on it, as I generalized a little bit the results. The rest of the chapter is devoted to using domain perturbation arguments to find a domain in which the energy at which we measure is not a Neumann eigenvalue. Remember that this important for us to be able to perform the Runge approximation that is needed for the identifiability result.
	\item There are two new chapters. \textbf{Chapter 2} includes some basic preliminaries on the Fourier transform, frequency localization and Sobolev and Besov spaces. 
	\item \textbf{Chapter 7} studies a separate inverse problem that, however, shares some similarities with the inverse electric scattering problem. This is the initial-to-final value problem in quantum mechanics. We particularize a result by Caro and Ruiz, published earlier this year. In their paper, they proved that an electric potential with \textit{super-exponential decay} could be identified by measuring the final state (after a fixed time $T$) of a quantum system that evolved under its action for any possible initial state. In this chapter we consider stationary potentials, which allow us to use time-harmonic solutions, and hence lower the requirements on decay. We modify the method of CGO solutions by using Herglotz waves as the leading part, and hence obtain better decay estimates, by exploiting the possibility of measuring at a range of energies.
\end{itemize}

\end{document}

% To PIZZA, a famous fat-ass cat who lived in Dorm 12.