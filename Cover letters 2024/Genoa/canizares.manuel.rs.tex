%%%%%%%%%%%%%%%%%%%%%%%%%%%%%%%%%%%%%%%%%
% Long letter 南京大学版 Nanjing University Version
% Version 1.0 (2022-03-27)

% This template was revised by Zheng-hu Nie(brian.nie@gmail.com) based on Fanchao Chen (chenfc@fudan.edu.cn).
% This template originates from:
% https://www.LaTeXTemplates.com

%%%%%%%%%%%%%%%%%%%%%%%%%%%%%%%%%%%%%%%%%

%----------------------------------------------------------------------------------------
%	PACKAGES AND OTHER DOCUMENT CONFIGURATIONS 文档基础配置
%----------------------------------------------------------------------------------------

\documentclass{article}

\usepackage{charter} % Use the Charter font

\usepackage[
	a4paper, % Paper size
	top=1in, % Top margin
	bottom=1in, % Bottom margin
	left=1in, % Left margin
	right=1in, % Right margin
	%showframe % Uncomment to show frames around the margins for debugging purposes
]{geometry}

\setlength{\parindent}{0pt} % Paragraph indentation
\setlength{\parskip}{1em} % Vertical space between paragraphs

\usepackage{graphicx} % Required for including images

\usepackage{hyperref}

\usepackage{fancyhdr} % Required for customizing headers and footers

\fancypagestyle{firstpage}{%
	\fancyhf{} % Clear default headers/footers
	\renewcommand{\headrulewidth}{0pt} % No header rule
	\renewcommand{\footrulewidth}{1pt} % Footer rule thickness
}

\fancypagestyle{subsequentpages}{%
	\fancyhf{} % Clear default headers/footers
	\renewcommand{\headrulewidth}{1pt} % Header rule thickness
	\renewcommand{\footrulewidth}{1pt} % Footer rule thickness
}

\AtBeginDocument{\thispagestyle{firstpage}} % Use the first page headers/footers style on the first page
\pagestyle{subsequentpages} % Use the subsequent pages headers/footers style on subsequent pages

%----------------------------------------------------------------------------------------

\begin{document}

%----------------------------------------------------------------------------------------
%	FIRST PAGE HEADER
%----------------------------------------------------------------------------------------


\vspace{-1em} % Pull the rule closer to the logo

\rule{\linewidth}{1pt} % Horizontal rule

%\bigskip\bigskip % Vertical whitespace

%----------------------------------------------------------------------------------------
%	YOUR NAME AND CONTACT INFORMATION
%----------------------------------------------------------------------------------------

\hfill
\begin{tabular}{l @{}}
\hfill \today \bigskip\\ % Date
\hfill Manuel Ca\~{n}izares \\
\hfill \href{mailto:mcanizares@bcamath.org}{mcanizares@bcamath.org}\\
\hfill BCAM (Basque Center for Applied Mathematics)\\
\hfill 48003. Bilbao, Spain \\ % Address
\end{tabular}

\bigskip % Vertical whitespace

%----------------------------------------------------------------------------------------
%	LETTER CONTENT
%----------------------------------------------------------------------------------------


In 2018, I obtained bachelor's degrees in both physics and mathematics through a double bachelor's program at the Universidad de Sevilla, Spain. I wrote my bachelor thesis while taking part in the Erasmus program at WWU M\"{u}nster, Germany. The thesis laid in the field of computational physics, I mainly conducted computer simulations to study phase transitions and the appearance of interfaces in the two dimensional Ising model with anti-periodic boundary conditions. This work resulted in a joint paper with my supervisor, Gernot M\"unster, published in 2021 in the \textit{Journal of Statistical Physics.}

I then went on to study a master's degree on mathematical physics at Universidad de Granada, Spain, from which I graduated in 2020. I decided to shift my focus towards mathematical analysis, driven by a strong interest in understanding crowd dynamics.  Consequently, I contacted Juan Soler, who supervised my thesis on the Strogatz model of swarmallators. This model aims to explain systems in which there is an interplay between synchronization and aggregation phenomena, from a kinetic point of view.

Following that, I had the opportunity to start a PhD under the supervision of Pedro Caro at the Basque Center for Applied Mathematics in Bilbao, Spain. I was able to choose between focusing either on inverse problems or on more pure harmonic analysis. I chose the first one, since I feel like I am greatly motivated by the study of physical phenomena through mathematics.

Prior to this experience, I had not been exposed to inverse problems. Therefore, my first year primarily focused on studying the background and techniques in analytical inverse problems. Pedro then proposed me that I extend a result that he and Andoni Garcia had published on 2020, on the identifiability of rough electric potentials via the near-field pattern in the context of electric scattering at fixed energies.

In November 2023, I submitted the result of my work as a paper for which I am the sole author. In this work, I obtained an inverse uniqueness result in a similar setting, in this case using local data. On the one hand, I used some harmonic analysis techniques to extend the class of potentials to consider. On the other hand, I proved Runge approximation and interior regularity results to make use of the local data to obtain a suitable orthogonality relation. The paper appeared in the journal \textit{Inverse Problems} in April 2024, and is a big part of my PhD thesis.

Between April and July 2024, I carried out an international stay at University of Bordeaux by invitation of Sylvain Ervedoza. During this stay, I worked under his supervision trying to fill a gap that appeared in the aforementioned work. This gap consisted in proving that one can always find a domain in which the energies at which one desires to measure are not Neumann eigenvalues for the operator $-\Delta+\lambda$. This is needed to perform the Runge approximation that is key in the solution of the problem, and I solved it using domain perturbation techniques. It has become a whole chapter of the thesis, and I am currently writing a paper on it.

Also, in the last year and a half, I have been working in the intial-to-final value inverse problem for the Schr\"odinger evolution equation. This setting was introduced by Pedro and Alberto Ruiz in a recent paper, and the idea is to identify electric potentials by measuring the final state of a quantum system that interacts with the potential, for every imposed initial state. Pedro and Alberto proved identifiability for bounded, \textit{super-exponentially} decreasing potentials.

One possibility is to relax the condition of super-exponential decay to just exponential decay. This relies on constructing an analytic family of special solutions -akin to CGO solutions- to the equation, following the work of Uhlmann and Vasy, and P\"aiv\"arinta, Salo and Uhlmann. I made remarkable progress on it, but the nature of the operator asks that we use a new local smoothing estimate to construct this family of solutions in the right spaces, and thus this work has been put on hold for now.

I have also been studying a second possibility in collaboration with Pedro and Ioannis Parissis, and more recently with Athanasios Zacharopoulos. This consists in considering time-independent potentials, which is connected with the previous scattering problem. We already obtained an uniqueness result for bounded potentials with polynomial decay, which I included in my thesis. We are currently working on extending this result to the case of potentials living in Lebesgue spaces of less regularity, which is showing to be a promising direction.

Having already sent my thesis to my committee, I am currently preparing for my defense, which is scheduled for mid-October, as well as writing the paper related to Neumann eigenvalues. I am eager to keep developing my career as a researcher. I am still interested in the use of harmonic analysis for inverse problems, and I hope to keep expanding my work in the future. For instance, I would like to show stability estimates for the problems I have been working on. However, I am greatly motivated by the idea of broadening my knowledge on the whole field, and exploring the possibilities that inverse problems offer in more applied settings, bridging the gaps between mathematics and other sciences.



\end{document}

% To PIZZA, a famous fat-ass cat who lived in Dorm 12.